% Options for packages loaded elsewhere
\PassOptionsToPackage{unicode}{hyperref}
\PassOptionsToPackage{hyphens}{url}
\documentclass[
  japanese,
]{article}
\usepackage{xcolor}
\usepackage{amsmath,amssymb}
\setcounter{secnumdepth}{-\maxdimen} % remove section numbering
\usepackage{iftex}
\ifPDFTeX
  \usepackage[T1]{fontenc}
  \usepackage[utf8]{inputenc}
  \usepackage{textcomp} % provide euro and other symbols
\else % if luatex or xetex
  \usepackage{unicode-math} % this also loads fontspec
  \defaultfontfeatures{Scale=MatchLowercase}
  \defaultfontfeatures[\rmfamily]{Ligatures=TeX,Scale=1}
\fi
\usepackage{lmodern}
\ifPDFTeX\else
  % xetex/luatex font selection
\fi
% Use upquote if available, for straight quotes in verbatim environments
\IfFileExists{upquote.sty}{\usepackage{upquote}}{}
\IfFileExists{microtype.sty}{% use microtype if available
  \usepackage[]{microtype}
  \UseMicrotypeSet[protrusion]{basicmath} % disable protrusion for tt fonts
}{}
\makeatletter
\@ifundefined{KOMAClassName}{% if non-KOMA class
  \IfFileExists{parskip.sty}{%
    \usepackage{parskip}
  }{% else
    \setlength{\parindent}{0pt}
    \setlength{\parskip}{6pt plus 2pt minus 1pt}}
}{% if KOMA class
  \KOMAoptions{parskip=half}}
\makeatother
\ifLuaTeX
\usepackage[bidi=basic,shorthands=off,provide=*]{babel}
\else
\usepackage[bidi=default,shorthands=off,provide=*]{babel}
\fi
\ifLuaTeX
  \usepackage{selnolig} % disable illegal ligatures
\fi
\setlength{\emergencystretch}{3em} % prevent overfull lines
\providecommand{\tightlist}{%
  \setlength{\itemsep}{0pt}\setlength{\parskip}{0pt}}
\usepackage{bookmark}
\IfFileExists{xurl.sty}{\usepackage{xurl}}{} % add URL line breaks if available
\urlstyle{same}
\hypersetup{
  pdftitle={エレギア式朗唱法},
  pdflang={ja},
  hidelinks,
  pdfcreator={LaTeX via pandoc}}

\title{エレギア式朗唱法}
\author{}
\date{}

\begin{document}
\maketitle

{
\setcounter{tocdepth}{3}
\tableofcontents
}
\begin{center}\rule{0.5\linewidth}{0.5pt}\end{center}

\protect\phantomsection\label{toc-root}

\begin{center}\rule{0.5\linewidth}{0.5pt}\end{center}

\subsection{序章}\label{chapter1}

\subsubsection{はじめに}\label{section1.1}

この朗唱法は、朗唱の基本的な概念と技術を学ぶためのリソースです。朗唱の歴史、種類、そして実践的なテクニックについて詳しく説明します。

\paragraph{朗唱の歴史}\label{subsection1.1.1}

朗唱は古代から存在し、宗教的儀式や文化的伝統の一部として発展してきました。様々な文化で朗唱は重要な役割を果たしており、その技術は時代とともに進化してきました。

\begin{center}\rule{0.5\linewidth}{0.5pt}\end{center}

\subsection{記号体系}\label{chapter2}

\subsubsection{朗唱記号}\label{section2.1}

朗唱記号は、音楽的なリズムや旋律を表現するために使用されます。それぞれの記号には特定の意味があり、朗唱のスタイルや感情を伝えるために使われます。

\paragraph{朗唱記号図表}\label{subsection2.1.1}

\subparagraph{朗唱記号表}\label{subsubsection2.1.1.1}

\subparagraph{朗唱記号図}\label{subsubsection2.1.1.2}

\begin{figure}
\centering
\caption{朗唱記号図}
\end{figure}

\paragraph{朗唱記号図表}\label{subsection2.1.2}

\begin{description}
\tightlist
\item[{◌́}]
高揚声符。
\item[{◌̗}]
低揚声符。
\item[{◌̀}]
高抑声符。
\item[{◌̖}]
低抑声符。
\item[{◌̂}]
高昇声符。
\item[{◌̭}]
低昇声符。
\item[{◌̌}]
高降声符。
\item[{◌̬}]
低降声符。
\item[{◌̄}]
高平声符。
\item[{◌̱}]
低平声符。
\item[{◌̈}]
高分声符。
\item[{◌̤}]
低分声符。
\item[{◌̇}]
高終声符。
\item[{◌̣}]
低終声符。
\end{description}

\begin{center}\rule{0.5\linewidth}{0.5pt}\end{center}

\subsection{発声と旋律体系}\label{chapter3}

\begin{center}\rule{0.5\linewidth}{0.5pt}\end{center}

\subsection{朗唱記号の統語規則}\label{chapter4}

\begin{center}\rule{0.5\linewidth}{0.5pt}\end{center}

\subsection{韻律と時間構造}\label{chapter5}

\begin{center}\rule{0.5\linewidth}{0.5pt}\end{center}

\subsection{言語構造と朗唱法}\label{chapter6}

\begin{center}\rule{0.5\linewidth}{0.5pt}\end{center}

\subsection{実修篇}\label{chapter7}

\begin{center}\rule{0.5\linewidth}{0.5pt}\end{center}

\subsection{典礼的用法}\label{chapter8}

\begin{center}\rule{0.5\linewidth}{0.5pt}\end{center}

\end{document}
